\section{Conclusion}
\label{sec:conclusion}
We presented MCRP, a decentralised cross-layer protocol with a centralised controller. Our protocol mitigates the effect of interference by avoiding affected channels. It allows better spectrum usage by trying to move nearby nodes to listen on different channels using two-hop colouring algorithm. Our protocol provides feedback when a channel is subject to interference using a probing phase.
The results from the simulation showed that our protocol avoids channels with interference hence greatly reduced loss rates with negligible overhead. By reducing packet loss (hence retransmissions) and increasing the efficiency of spectrum usage, the multichannel system will be more energy efficient than single channel ContikiMAC with RPL over the lifetime of the system's deployment.

Future work is ongoing to develop the protocol. Deployment is underway on the Flocklab testbed \cite{flocklab}. Next we plan to improve the interference model we used to better replicate the real world environment. 
%The protocol will be tested against competing multi-channel protocols such as MiCMAC. We also plan to test our implementation on real hardware.  Finally we will allow nodes to update the LPBR on ongoing packet loss so that the network can continually respond to changes in congestion.

